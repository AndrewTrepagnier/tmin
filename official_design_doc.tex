\documentclass[11pt]{article}

%--------------------------------------------------
% Packages
%--------------------------------------------------
\usepackage[a4paper,margin=1in]{geometry}
\usepackage{graphicx}
\usepackage{amsmath, amssymb}
\usepackage{hyperref}
\usepackage{fancyhdr}
\usepackage{color}
\usepackage{listings}
\usepackage{float}
\usepackage{titlesec}
\usepackage{setspace}
\usepackage{lmodern}
\usepackage{titling}

%--------------------------------------------------
% Metadata
%--------------------------------------------------
\newcommand{\doctitle}{TMIN: The Fast Pipe Thickness Analysis Tool}
\newcommand{\docsubtitle}{A Technical Design Document}
\newcommand{\authorname}{Andrew Trepagnier}
\newcommand{\organization}{Author Affiliations: ExxonMobil Corporation \\ Mechanical Integrity Engineering Group, BMRF}

%--------------------------------------------------
% Code Listings Style
%--------------------------------------------------
\lstset{
  basicstyle=\ttfamily\small,
  backgroundcolor=\color{gray!10},
  frame=single,
  breaklines=true,
  captionpos=b,
  numbers=left,
  numberstyle=\tiny,
  keywordstyle=\color{blue},
  commentstyle=\color{gray},
  stringstyle=\color{red},
  showstringspaces=false
}

%--------------------------------------------------
% Header & Footer
%--------------------------------------------------
\pagestyle{fancy}
\fancyhf{}
\rhead{TMIN Design Document}
\lhead{}
\rfoot{Page \thepage}

%--------------------------------------------------
% Title Formatting
%--------------------------------------------------
\titleformat{\section}
  {\normalfont\Large\bfseries}{\thesection}{1em}{}

\titleformat{\subsection}
  {\normalfont\large\bfseries}{\thesubsection}{1em}{}

%--------------------------------------------------
% Begin Document
%--------------------------------------------------
\begin{document}

%--------------------------------------------------
% Cover Page
%--------------------------------------------------
\begin{titlepage}
    \centering
    \vspace*{4cm}
    {\Huge \bfseries \doctitle \par}
    \vspace{0.5cm}
    {\Large \docsubtitle \par}
    \vfill
    {\large \today \par}
    \vspace{14cm}
    \rule{\linewidth}{0.5pt}
    \vspace{0.3cm}
    {\small \authorname \par}
    {\small \organization \par}
\end{titlepage}

%--------------------------------------------------
% Table of Contents
%--------------------------------------------------
\tableofcontents
\newpage

%--------------------------------------------------
% Sections
%--------------------------------------------------
\section{Purpose}
The purpose of this document is to provide users with the mathematical approaches and procedures used within the computations of the open source software package, TMIN. After reading, users should have a thorough understanding of how minimum thickness calculations are determined according to industry code as well as what the TMIN package's limitations are.


\section{What is TMIN}

TMIN is an abbreviated term for "minimum thickness" within the process piping inspection and engineering. Minimum thickness is a pipe's lowest allowable wall thickness before mechanical/structural failure, such as bursting or folding under it's own weight, becomes a hazard. Most process pipes experience some degree of internal corrosion, thus, causing thinning over time and forcing pipe retirement before a potential loss of containment.

\section{Scientific Basis}
TMIN is based on established mechanical engineering principles from ASME B31.3 and API 574. It evaluates hoop stress, corrosion allowances, and pressure limits based on user-defined material and geometry parameters.

Thinned pipes can fail structurally or due to pressure. Section 3.1 discusses the mathematical approach to solving the minimum allowable pressure-containing thickness and Section 3.2 discusses structural considerations, followed by Section 4 and 5, the limits and assumptions of the TMIN package. 

\subsection{Minimum Pressure-Containing Thickness }
\begin{equation}
    t_{\text{min}} = \frac{P D}{2( S E W + P Y)} 
\end{equation}

Where:
\begin{itemize}
    \item $P$ = Design Pressure
    \item $D$ = Outside Diameter
    \item $S$ = Allowable Stress
    \item $E$ = Weld Joint Efficiency
    \item $Y$ = Coefficient for Material
    \item $C$ = Corrosion Allowance
\end{itemize}

\section{Core Features}
\begin{itemize}
    \item Simple CLI and API interface
    \item Standards-based calculations (ASME)
    \item Fast performance for batch evaluations
    \item Extensible and modular Python codebase
    \item Includes error-checking and input validation
\end{itemize}

\section{Installation and Usage}
\subsection*{Installation}
\begin{lstlisting}[language=bash]
pip install tmin
\end{lstlisting}

\subsection*{Example Usage}
\begin{lstlisting}[language=Python]
from tmin import calculate_tmin

t = calculate_tmin(P=150, D=10.75, S=20000, E=1.0, Y=0.4, C=0.125)
print(f"Minimum thickness: {t:.3f} in")
\end{lstlisting}

\section{Limitations and Assumptions}
\begin{itemize}
    \item Assumes linear elastic material behavior
    \item Only applicable to cylindrical geometries
    \item Requires accurate material properties from user
    \item Not a substitute for full FEA or regulatory analysis
\end{itemize}

\section{Design Decisions}
\begin{itemize}
    \item Written in pure Python for accessibility
    \item No external dependencies for core functions
    \item Open-source with permissive license
    \item Design favors speed over generality
\end{itemize}

\section{Use Cases}
\begin{itemize}
    \item Rapid design checks during piping layout
    \item Automation of pipe thickness reports
    \item Educational tool for engineering students
    \item Batch processing of pipe specifications
\end{itemize}

\section{Future Work}
\begin{itemize}
    \item Add support for additional design codes (e.g., B31.1, EN)
    \item Build a web front-end for visualization
    \item Add sensitivity analysis / uncertainty quantification
    \item Export capabilities to Excel and PDFs
\end{itemize}

\section{References}
\begin{itemize}
    \item ASME B31.3 – Process Piping Code
    \item Roark's Formulas for Stress and Strain
    \item Python Software Foundation Documentation
\end{itemize}

\end{document}
